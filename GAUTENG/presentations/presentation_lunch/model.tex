\documentclass[12pt]{article}

% TEMPLATE DEFAULT PACKAGES
\usepackage{amssymb,amsmath,amsfonts,eurosym,geometry,ulem,graphicx,caption,color,setspace,sectsty,comment,footmisc,caption,natbib,pdflscape,subfigure,array,hyperref}

% ADDED PACKAGES FOR THIS MANUSCRIPT
\usepackage{mathptmx,multirow,endfloat,titlesec,threeparttable,mathtools,bm,bbm}

% SECTION TITLE SETTINGS
\titlelabel{\thetitle.\enskip}
\titleformat*{\section}{\large\bfseries}
\titleformat*{\subsection}{\normalsize\bfseries}

% COLUMN TYPES
\newcolumntype{L}[1]{>{\raggedright\let\newline\\\arraybackslash\hspace{0pt}}m{#1}}
\newcolumntype{C}[1]{>{\centering\let\newline\\\arraybackslash\hspace{0pt}}m{#1}}
\newcolumntype{R}[1]{>{\raggedleft\let\newline\\\arraybackslash\hspace{0pt}}m{#1}}

% MARGINS AND SPACING
\normalem
\onehalfspacing
\geometry{left=1.0in,right=1.0in,top=1.0in,bottom=1.0in}

% SPECIAL CELL 
\newcommand{\specialcell}[2][c]{%
	\begin{tabular}[#1]{@{}l@{}}#2\end{tabular}}

\begin{document}

\begin{center}
{\large \bf  \centering \noindent A Model of Housing with Slum Externalities}
\end{center}

\subsection*{Basic Setup}

\begin{itemize}
\setlength\itemsep{-.5em}
\setlength\itemindent{-15pt}
	\item A city is comprised of $N$ residential neighborhoods indexed $n\in\{1,...,N\}$. 
	\item Each neighborhood has a formal and informal housing sector, indexed $s\in\{F,I\}$.
	\item All jobs are in the CBD and pay a fixed wage rate; commuting to the CBD is costly.
\end{itemize}

\subsection*{Housing Demand}

\noindent Measure one of freely-mobile agents inelastically demand one unit of housing. An agent $i$ has idiosyncratic tastes for neighborhoods, represented by the vector $\bm{\epsilon}_i \in \mathbb{R}^{2N}$ of neighborhood-sector specific valuations. The distribution of preferences in the population is given by the continuous and well-behaved function $f(\bm{\epsilon})$ such that $\int f(\bm{\epsilon}) d\bm{\epsilon} = 1$.
The indirect utility of individual $i$ living in sector $s$ of neighborhood $n$ is given by:

\begin{equation*}
\begin{aligned}
u_{ins} \,& =\, A_{ns}\,-\, R_{ns} \,+\, \epsilon_{ins} \\
        \,& =\, v_{ns} \,+\, \epsilon_{ins}
\end{aligned}
\end{equation*}

\noindent where $A_{ns}$ may be interpreted as the net-of-commuting and amenity-adjusted wage received in $n$ when living in sector $s$, $R_{ns}$ is the rental rate of housing, and $\epsilon_{ins}$ is individual $i$'s idiosyncratic taste for neighborhood-sector pair $(n,s)$. For notational convenience, we denote the array of quantities $A_{ns}$, $R_{ns}$ and $\epsilon_{ins}$ in their matrix form as $\bm{A}$, $\bm{R}$ and $\bm{\epsilon}_i$. Individual $i$ takes exogenous quantities $\{\bm{A},\bm{\epsilon}_i\}$ and endogenous rents $\bm{R}$ as given, and locates in pair $(n,s)$ yielding the highest indirect utility. Aggregate housing demand $H_{ns}$ in $(n,s)$ is given by:

\begin{equation*}
\begin{aligned}
H_{ns} \,\,& =\,\, \int I\Big( \, u_{ins} \,=\, \max_{n's'}\{u_{in' s'} \,\,|\,\, \bm{A},\bm{R}\,\} \Big)f(\bm{\epsilon}) d\bm{\epsilon}    \\[.2em]
        \,\,& =\,\, D_{ns}(\bm{A},\bm{R})
\end{aligned}
\end{equation*}

\subsection*{Housing Supply}

\noindent Neighborhoods are endowed with $L_n$ units of land. A share $\theta_{nF}$ of the total land stock is available for formal residential development, while the rest, $\theta_{nI} = 1-\theta_{nF}$, is vacant or public land suited for informal housing. We assume that $\theta_{nF}$ is determined exogenously. In each sector, price-taking landlords supply housing by combining land and materials $M$ with CRS technology $q(L,M)$. Materials are supplied at fixed cost $c$ on a large national market. Because land is fixed in every market, the marginal cost of housing is increasing. The government may also choose to subsidize housing in market $(n,s)$ at a rate of $\delta_{ns}$ per unit. The inverse supply curve in $(n,s)$ is:

\begin{equation*}
R_{ns} \,\, =\,\, S(H_{ns},\theta_{ns}L_n) - \delta_{ns}
\end{equation*} \\[-1.99em]

\noindent where $\partial S/\partial H > 0$ and $\partial S/\partial L < 0$. For notational convenience, we denote the array of quantities $H_{nh}$, $L_n$ and $\theta_n$ in their vector form as $\bm{H}$, $\bm{L}$ and $\bm{\theta}$, respectively.

\subsection*{Equilibrium}

Given density $f(\bm{\epsilon})$ and the exogenous quantities $\{\bm{A},\bm{L},\bm{\theta}\}$, an equilibrium in this city consists of rents $\bm{R}^*$ and housing quantities $\bm{H}^*$ such that all housing markets clear, that is, $\forall (n,h)$:

\begin{equation}
\begin{aligned}
& H^*_{ns} \,\, =\,\, D_{ns}(\bm{A}, \bm{R}^*) \\
\text{and}\quad & R^*_{ns} \,\, =\,\, S(H^*_{ns},\theta_{ns}L_n) - \delta_{ns} \text{,}
\end{aligned}
\end{equation}\\[-1.99em]\\[-1.99em]

\noindent and all agents live somewhere:

\begin{equation*}
\sum_{n}\sum_{s} H^*_{ns} = 1 
\end{equation*} \\[-1.99em]

\subsection*{Welfare Implications of Housing Subsidies}

\noindent The sum of individuals' utility is given by:

\begin{equation*}
U \,\, =\,\, \int \max_{n's'}\{u_{in' s'}\,\,|\,\, \bm{A},\bm{R}^*\,\}f(\bm{\epsilon}) d\bm{\epsilon} 
\end{equation*} \\[-1.99em]

\noindent and total landlords' profit is:

\begin{equation*}
\begin{aligned}
\Pi \,\, &=\,\, \sum_{n}\sum_{s} \Bigg(\,\int_0^{H^*_{ns}} [R^*_{ns} - (S(x,\theta_{ns}L_n) - \delta_{ns})]dx \, \Bigg) \\[.6em]
\,\, &=\,\, \sum_{n}\sum_{s} \Bigg(\, R^*_{ns}H^*_{ns} +  \delta_{ns}H^*_{ns} -  \int_0^{H^*_{ns}}S(x,\theta_{ns}L_n)dx  \,\Bigg)
\end{aligned}
\end{equation*} \\[-1.99em]

\noindent Total social welfare in this economy is therefore $W = U + \Pi$. We are interested in the welfare implications of the government subsidizing formal housing in a subset $\mathcal{N}\subset\{1,...,N\}$ of neighborhoods. Let $\delta_{ns}=\delta$ if $n\in\mathcal{N}$ and $s=F$, and $\delta_{ns}=0$ otherwise.  To characterize the marginal social benefit from the subsidy, $\partial W/\partial \delta$,  we first note a result shown in Busso et al. (2013):

\begin{equation*}
\begin{aligned}
\frac{\partial U}{\partial v_{ns}} \,\, & =\,\, \int \frac{\partial }{\partial v_{ns}} \max_{n's'}\{u_{in' s'}\,\,|\,\, \bm{A},\bm{R}\,\}f(\bm{\epsilon}) d\bm{\epsilon} \\[.6em]
\,\, & =\,\, \int I\Big( \, u_{ins} \,=\, \max_{n's'}\{u_{in' s'} \,\,|\,\, \bm{A},\bm{R}\,\} \Big)f(\bm{\epsilon}) d\bm{\epsilon} \\[.6em]
\,\, & =\,\  H_{ns}
\end{aligned}
\end{equation*} \\[-1.99em]

\noindent Using the above, we write the derivative of total utility with respect to subsidy $\delta$ as:

\begin{equation*}
\frac{\partial U}{\partial\delta} \,\,=\,\,  \sum_{n}\sum_{s}\,\, H^*_{ns}\Big(-\frac{\partial R_{ns}^*}{\partial \delta}\Big) 
\end{equation*} \\[-1.99em]

\noindent The derivative of total profits with respect to $\delta$ is:

\begin{equation*}
\frac{\partial \Pi}{\partial\delta} \,\,=\,\, \sum_{n\in\mathcal{N}} H^*_{nF} \,\,+\,\, \sum_{n}\sum_{s}\, \Big[ H^*_{ns}\frac{\partial R^*_{ns}}{\partial \delta} \,\,+\,\, R^*_{ns}\frac{\partial H^*_{ns}}{\partial \delta} \,\,+\,\, \delta_{ns}\frac{\partial H^*_{ns}}{\partial \delta} \,\,-\,\, \frac{\partial H^*_{ns}}{\partial \delta}\big( R^*_{ns} \,\, + \,\, \delta_{ns} \big) \Big]
\end{equation*} \\[-1.99em]

\noindent Summing and simplifying:

\begin{equation}
\frac{\partial W}{\partial\delta} \,\,=\,\, \frac{\partial U}{\partial\delta} \,\,+\,\, \frac{\partial \Pi}{\partial\delta} \,\,=\,\, \sum_{n\in\mathcal{N}} H^*_{nF}
\end{equation} \\[-1.99em]

\noindent The total cost of the subsidy is given by $TC = \sum_n\sum_s H^*_{ns}\delta_{ns}$ and its marginal cost is thus:
\begin{equation*}
 \frac{\partial TC}{\partial \delta} \,\,=\,\, \sum_{n\in\mathcal{N}} \Big( H^*_{nF} \,\,+\,\, \delta\frac{\partial H^*_{nF}}{\partial \delta} \Big)
 \end{equation*} \\[-1.99em]

 \noindent The extra term in this expression relative to (2) represents the marginal deadweight loss from an increase in $\delta$. Unsurprisingly, we find that subsidies in an economy with perfect housing markets lead to economic inefficiencies. Furthermore, the magnitude of these inefficiencies depends critically on the population responses in subsidized markets.

 \subsection*{Housing subsidies with Slum Externalities}

 Densely populated slums pose fire hazards, increase health risks and overburden existing public infrastructure, e.g. water and sewage networks. We consider an external utility cost from informal housing of the form:

\begin{equation*}
A_{ns} \,\,=\,\, \bar{A}_{ns} \,\,+\,\, a\Big(\frac{H_{nI}}{L_n}\Big)
\end{equation*} \\[-1.99em]

\noindent where $a'(.)<0$. With this specification, the private decision of locating in $(n,I)$ negatively impacts all residents in $n$ because of congestion effects. The utility response to $\delta$ now depends on how both rents and amenities change in equilibrium:

\begin{equation*}
\frac{\partial U}{\partial\delta} \,\,=\,\,  \sum_{n}\sum_{s} \,\, H^*_{ns}\Big(a'\Big(\frac{H^*_{nI}}{L_n}\Big)\frac{\partial H^*_{nI}}{L_n\partial \delta}\,\,-\,\,\frac{\partial R_{ns}^*}{\partial \delta}\Big) 
\end{equation*} \\[-1.99em]

\noindent Therefore:

\begin{equation*}
\frac{\partial W}{\partial\delta} \,\,=\,\, \sum_{n\in\mathcal{N}} H^*_{nF} \,\,+\,\, \sum_{n} a'\Big(\frac{H^*_{nI}}{L_n}\Big)\,\frac{\partial H^*_{nI}}{\partial \delta}\,\frac{(H^*_{nF}+H^*_{nI})}{L_n}
\end{equation*} \\[-1.99em]

\noindent Combining the equilibrium conditions in (1) and using the implicit function theorem, it is possible to show formally that $\partial H^*_{nI}/\partial \delta < 0 \,\,\forall n$. The extra term in the above expression -- relative to equation (2) -- represents the marginal welfare {\it gain} from reduced slum density. The subsidy $\delta$ makes formal housing in $\mathcal{N}$ more attractive relative to informal housing. Marginal residents moving to $\mathcal{N}$ make remaining residents better-off because of reduced congestion. The marginal deadweight loss (MDWL) of the subsidy is now :

\begin{equation*}
\begin{aligned}
MDWL \,\,&=\,\, \frac{\partial TC}{\partial \delta} - \frac{\partial W}{\partial \delta} \\[.6em]
     \,\,&=\,\, \sum_{n\in\mathcal{N}} \,\delta\frac{\partial H^*_{nF}}{\partial \delta} \,\,-\,\, \sum_{n} \, a'\Big(\frac{H^*_{nI}}{L_n}\Big)\,\frac{\partial H^*_{nI}}{\partial \delta}\,\frac{(H^*_{nF}+H^*_{nI})}{L_n} \\
\end{aligned}
\end{equation*} \\[-1.99em]

\noindent Efficiency considerations in this setting depend on the population responses in subsidized markets, but also on responses in informal markets and on the shape of slum externality $a(\,)$. We note that $\delta=0$ implies $MDWL<0$, that is, some level of subsidy $\delta$ is welfare improving when compared to no subsidy. This is expected since the social benefits exceed the private benefits of moving from informal to formal housing. 

\subsection*{Subsidy Spillovers}

In the context of South Africa's RDP program, the delivery of subsidized formal housing provides opportunities for "backyarding", a prevalent phenomena where informal shack dwellers settle in the backyard of formal structures. Since the policy effectively lowers construction costs for informal housing, we model this by assuming subsidies in the formal sector spillover to the informal sector at no additional cost for the government. Formally, we let $\delta_{ns} = \delta$ if $n\in\mathcal{N}$ and $s=F$, $\delta_{ns} = \alpha\delta$ if $n\in\mathcal{N}$ and $s=I$, and $\delta_{ns} = 0$ otherwise, where $\alpha\in\mathbb{R}^+$. The presence of such spillovers alter welfare implications in two ways. First, suppliers of informal housing benefit from an increase in profit due to the indirect subsidies. Second, for $n\in\mathcal{N}$, the sign of informal housing response $\frac{\partial H^*_{nI}}{\partial\delta}$ is now ambiguous and depends on the magnitude of $\alpha$.\footnote{ This can also be shown formally by combining the equilibrium conditions in (1) and using the implicit function theorem. For $n\in\mathcal{N}^C$, $\frac{\partial H^*_{nI}}{\partial\delta}$ remains unambiguously negative.} We use the notation $\frac{\partial H^*_{nI}}{\partial\delta}(\alpha)$ to represent this dependence. When $\alpha$ is large, the indirect subsidies in $(n,I)$ dominate and net-migration is positive, i.e. $\frac{\partial H^*_{nI}}{\partial\delta}>0$, making incumbent residents in $(n,I)$ and $(n,F)$ worse-off. With similar derivations as above, we obtain:

\begin{equation*}
\frac{\partial W}{\partial\delta} \,\,=\,\, \sum_{n\in\mathcal{N}} \big( H^*_{nF} + \alpha H^*_{nI} \big)  \,\,+\,\, \sum_{n} a'\Big(\frac{H^*_{nI}}{L_n}\Big)\,\frac{\partial H^*_{nI}}{\partial \delta}(\alpha)\,\Big(\frac{H^*_{nF}+H^*_{nI}}{L_n} \Big)
\end{equation*} \\[-1.99em]

\noindent and therefore:

\begin{equation*}
MDWL \,\,=\,\, \sum_{n\in\mathcal{N}} \,\delta\frac{\partial H^*_{nF}}{\partial \delta} \,\,-\,\, \sum_{n\in\mathcal{N}} \alpha H^*_{nI}\,\,-\,\, \sum_{n} \, a'\Big(\frac{H^*_{nI}}{L_n}\Big)\,\frac{\partial H^*_{nI}}{\partial \delta}(\alpha)\,\Big(\frac{H^*_{nF}+H^*_{nI}}{L_n} \Big)
\end{equation*} \\[-1.99em]

\subsection*{Takeaways}

\noindent This model highlights key quantities to be estimated in order to assess the incidence and efficiency of formal housing subsidies. Specifically, welfare considerations depend critically on the magnitude of spillovers $\alpha$, the shape of slum externalities $a(\,)$, and the population (housing) responses in both subsidized formal markets, $\frac{\partial H^*_{nF}}{\partial \delta}\,\,\forall n\in\mathcal{N}$,  and informal markets, $\frac{\partial H^*_{nI}}{\partial \delta}\,\,\forall n$. Empirical analysis should therefore focus on obtaining reliable estimates of theses quantities.





\end{document}