%%%%%%%%%%%%%%%%%%%%%%%%%%%%%%%%%%%%%%%%%
% Beamer Presentation
% LaTeX Template
% Version 1.0 (10/11/12)
%
% This template has been downloaded from:
% http://www.LaTeXTemplates.com
%
% License:
% CC BY-NC-SA 3.0 (http://creativecommons.org/licenses/by-nc-sa/3.0/)
%
%%%%%%%%%%%%%%%%%%%%%%%%%%%%%%%%%%%%%%%%%

%----------------------------------------------------------------------------------------
%	PACKAGES AND THEMES
%----------------------------------------------------------------------------------------

\documentclass[aspectratio=149]{beamer}
\usefonttheme[onlymath]{serif}


\mode<presentation> {

% The Beamer class comes with a number of default slide themes
% which change the colors and layouts of slides. Below this is a list
% of all the themes, uncomment each in turn to see what they look like.

\usetheme{default}
%\usetheme{AnnArbor}
%\usetheme{Antibes}
%\usetheme{Bergen}
%\usetheme{Berkeley}
%\usetheme{Berlin}
%\usetheme{Boadilla}
%\usetheme{CambridgeUS}
%\usetheme{Copenhagen}
%\usetheme{Darmstadt}
%\usetheme{Dresden}
%\usetheme{Frankfurt}
%\usetheme{Goettingen}
%\usetheme{Hannover}
%\usetheme{Ilmenau}
%\usetheme{JuanLesPins}
%\usetheme{Luebeck}
%\usetheme{Malmoe}
%\usetheme{Marburg}
%\usetheme{Montpellier}
%\usetheme{PaloAlto}
%\usetheme{Pittsburgh}
%\usetheme{Rochester}
%\usetheme{Singapore}
%\usetheme{Szeged}
%\usetheme{Warsaw}

% As well as themes, the Beamer class has a number of color themes
% for any slide theme. Uncomment each of these in turn to see how it
% changes the colors of your current slide theme.

%\usecolortheme{albatross}
\usecolortheme{beaver}
%\usecolortheme{beetle}
%\usecolortheme{crane}
%\usecolortheme{dolphin}
%\usecolortheme{dove}
%\usecolortheme{fly}
%\usecolortheme{lily}
%\usecolortheme{orchid}
%\usecolortheme{rose}
%\usecolortheme{seagull}
%\usecolortheme{seahorse}
%\usecolortheme{whale}
%\usecolortheme{wolverine}

%\setbeamertemplate{footline} % To remove the footer line in all slides uncomment this line
%\setbeamertemplate{footline}[page number] % To replace the footer line in all slides with a simple slide count uncomment this line

%\setbeamertemplate{navigation symbols}{} % To remove the navigation symbols from the bottom of all slides uncomment this line
}

\usepackage{graphicx} % Allows including images
\usepackage{booktabs} % Allows the use of \toprule, \midrule and \bottomrule in tables
\usepackage{verbatim}

\usepackage{mathtools} 
\usepackage{amssymb}
\usepackage{mathrsfs}
\usepackage{amsmath}
\usepackage{bm}

\usepackage{ragged2e}
\usepackage{etoolbox}
\usepackage{lipsum}

\usepackage{siunitx,booktabs}
\usepackage{pifont}
\usepackage{array}
\usepackage{tabu,booktabs}
\usepackage{tikz}
\usetikzlibrary{arrows,shapes}

\setbeamertemplate{enumerate items}[circle]
\usepackage{tikz}

\newcommand\mynum[1]{
  \usebeamercolor{enumerate item}
  \tikzset{beameritem/.style={circle,inner sep=0,minimum size=2ex,text=enumerate item.bg,fill=enumerate item.fg,font=\footnotesize}}%
  \tikz[baseline=(n.base)]\node(n)[beameritem]{#1};
}

\newcommand\mynumm[1]{
  \usebeamercolor{enumerate item}
  \tikzset{beameritem/.style={rectangle,inner sep=0,minimum size=2ex,text=enumerate item.bg,fill=enumerate item.fg,font=\footnotesize}}%
  \tikz[baseline=(n.base)]\node(n)[beameritem]{#1};
}

\def\Put(#1,#2)#3{\leavevmode\makebox(0,0){\put(#1,#2){#3}}}

\setbeamertemplate{footline}[]

%----------------------------------------------------------------------------------------
%	TITLE PAGE
%----------------------------------------------------------------------------------------

\title{ Subsidized Housing with Slum Externalities: \\ Evidence from South Africa } % The short title appears at the bottom of every slide, the full title is only on the title page

\author{Stefano Polloni \\
joint with Ben Bradlow and Will Violette} 

 % Your institution as it will appear on the bottom of every slide, may be shorthand to save space

\date{February 2017} %\today} % Date, can be changed to a custom date

\begin{document}

\beamertemplatenavigationsymbolsempty

\begin{frame}
\titlepage % Print the title page as the first slide
\end{frame}

%\begin{frame}
%\frametitle{Overview} % Table of contents slide, comment this block out to remove it
%\tableofcontents % Throughout your presentation, if you choose to use \section{} and \subsection{} commands, these will automatically be printed on this slide as an overview of your presentation
%\end{frame}

%----------------------------------------------------------------------------------------
%	PRESENTATION SLIDES
%----------------------------------------------------------------------------------------
\section{Introduction}
%------------------------------------------------

\begin{frame}
\frametitle{Slums and Development}

% In developing countries, 30\% of urban pop live in slums (UN, 2015)

\begin{itemize}
  \item Slum externalities $\rightarrow$ lasting poverty traps (Marx, 2013) 
    \begin{itemize}
      \item Poor infrastructure, high crime, health externalities
      \item Weak incentives to invest in housing/public goods
    \end{itemize}  
\vspace{.2cm}
\pause
  \item \textbf{Public Housing} $\rightarrow$ primary government response
\end{itemize}
    \pause
\begin{enumerate}
  \item Direct Recipient Impacts
    \begin{itemize}
      \item Health, Wellbeing, Employment, Redistribution \\ \footnotesize{(Cateneo et al. [2009], Franklin et al. [2016], Galiani et al. [2017])}
    \end{itemize}
% ``key strategy for poverty alleviation'' {\footnotesize{South Africa}}
    \pause
    \vspace{.1cm}
\item Neighborhood Development
  \begin{itemize}
    \item ``combating crime, promoting social cohesion... spatial restructuring'' South Africa Dept. of Human Settlements
    \pause
    \item Little research on spillovers \footnotesize{(Diamond and McQuade (2016))}
  \end{itemize}
\end{enumerate}

\begin{itemize}
  \item \textbf{Question} \\ 
  \vspace{.1cm}
  What are the spillovers from public housing in developing contexts?
\end{itemize}
% ``combating crime, promoting social cohesion... spatial restructuring,'' {\footnotesize{South Africa}}
\end{frame}

%------------------------------------------------

\begin{frame}
\frametitle{This Paper}

\begin{itemize}
  \item \textbf{Question} \\ 
  \vspace{.1cm}
  What are the spillovers from public housing in developing contexts?
  \vspace{.1cm}
    \begin{itemize}
      \item \textbf{Positive:} Incentivize investments in housing/public goods
      \item \textbf{Negative:} Crowd in slum growth
    \end{itemize}

\pause
\vspace{1mm}
\item \textbf{Approach} \\ Leverage precise timing/geography of large housing projects
\vspace{1mm}
\item \textbf{New Data and Setting} \\ 172 projects in South Africa combined with GPS property transactions and slum growth data
\vspace{1mm}
\item \textbf{Initial Findings} \\ Housing projects depress home prices by 5\% within 300 meters
\begin{itemize}
  \item heterogeneity
  \item ballpark estimates
\end{itemize}
\end{itemize}

\end{frame}


%------------------------------------------------

\begin{frame}
\frametitle{Public Housing in South Africa}
  \begin{itemize}
      \item Over 4.3 million houses since 1994 (13\% of pop.)
%      \item Houses 13\% of the population
%      \item Single-story, two-room (30-40$\text{m}^2$) dwellings
      \begin{itemize}
        \item 50 to 500 houses per project
      \end{itemize}
  \end{itemize}
    %%% insert picture


\begin{itemize}
        \item Who gets a house?
      \begin{itemize}
        \item Official Policy: 
          \begin{itemize}
            \item National/provincial waiting lists
            \item No resale within 7 years
            \item Citizens, new homeowners, married or dependents, inc/month $<$R3,500
          \end{itemize}
        \item In Practice:
          \begin{itemize}
            \item Waiting lists/eligibility weakly enforced
            \item Only 82\% of houses occupied by initial owners within 5 yrs
          \end{itemize}
      \end{itemize}
\end{itemize}
\end{frame}

%------------------------------------------------

\begin{frame}
\frametitle{Where are these houses built?}

\begin{enumerate}
  \item \textbf{Greenfield projects} on undeveloped land near slums
  \item \textbf{In-Situ upgrading} replacing existing slums
\end{enumerate}

\begin{itemize}
  \item insert picture here
\end{itemize}

\begin{itemize}
  \item Projects are fully serviced (roads, water, sanitation, electricity)
\end{itemize}
    %%% insert small maps of pre-post periods (with cherry picked examples) (with google maps? or bblu?) or find good sattelite from an evaluation?

\end{frame}

%------------------------------------------------



\begin{frame}
\frametitle{Conceptual Framework: Public Housing Impacts}

\begin{enumerate}
  \item \textbf{Amenity Effect:} Upgrading housing stock/services
    \begin{itemize}
        \item Increase value of neighboring homes (Rossi-Hansberg [2010])
    \end{itemize}
\vspace{.2cm} 

  \item \textbf{Crowd-In Slums:} Reduce costs of informal housing
    \begin{itemize}
        \item Overburden services, health/crime externalities
        \item Reduce value of nearby houses
    \end{itemize}
\vspace{.2cm}

  \item \textbf{Demographic Effect:} New people in the neighborhood
    \begin{itemize}
        \item Taste-based discrimination (Diamond and McQuade [2016])
    \end{itemize}
\end{enumerate}

%\begin{enumerate}
%  \item Housing externalities  {\small (Rossi-Hansberg [2010]) }
%    \begin{itemize}
%      \item Home value depends on the quality of neighboring homes
%      \item ie. informal dwellings have poor sanitation
%    \end{itemize}
%\vspace{.1cm}
%  \item Demographic externalities
%    \begin{itemize}
%      \item Taste-based discrimination
%      \item Overcrowding $\rightarrow$ crime/health externalities
%    \end{itemize}
%\vspace{.1cm}
%  \item Access to public goods (water, sewer, road infrastructure)
%\end{enumerate}

\end{frame}



%------------------------------------------------



\begin{frame}
\frametitle{Measuring Public Housing and Spillovers}

\begin{itemize}
  \item Focus on Gauteng Province (includes Johannesburg and Pretoria)
\end{itemize}

\begin{enumerate}
\item \textbf{Property Transactions} measure housing projects and price impacts
  \begin{itemize}
    \item 500,000 deeds records (bottom 20\% of formal housing market)
    \item Buyer/seller name, GPS, price, date from 2002-2011
  \end{itemize}
\vspace{.2cm}
\item \textbf{Building Census} identifies slum-growth and in-situ upgrading
  \begin{itemize}
    \item 4 mil. residential buildings (50\% informal) GPS in 2001 and 2011
  \end{itemize}
\vspace{.2cm}
\item \textbf{Population Census} measures demographic and economic impacts
  \begin{itemize}
    \item Full census for 18,000 census blocks in 2001 and 2011
  \end{itemize}
\vspace{.2cm}
\item \textbf{Administrative Data} map projects (construction dates and costs)
  \begin{itemize}
    \item Not comprehensive
    \item Includes planned but unconstructed projects
  \end{itemize}


\end{enumerate}

\end{frame}

%------------------------------------------------

%  
%  
%  \item \textbf{Temporal Clustering:} include cluster with $>$50\% of transactions during modal year


\begin{frame}
\frametitle{Identifying Housing Projects}

\begin{tikzpicture}[overlay]

\onslide<1->\node[overlay,anchor=west,align=left] at (0, 2.5) {    \begin{minipage}{1\textwidth} {\begin{enumerate}
  \item \textbf{Seller Identity:} match government names and housing authorities in seller-names from transactions 
\end{enumerate}}\end{minipage}};

\onslide<2->\node[overlay,anchor=west,align=left] at (0, 1.5) {    \begin{minipage}{1\textwidth} { 
\begin{enumerate}
  \setcounter{enumi}{1}
  \item \textbf{Subsidy Value:} exclude purchase prices R50,000 above subsidy value
\end{enumerate}}\end{minipage}};

\onslide<3->\node[overlay,anchor=west,align=left] at (0, .5) {    \begin{minipage}{1\textwidth} { 
\begin{enumerate}
  \setcounter{enumi}{2}
  \item \textbf{Pre-Existing Formal Dwellings:} exclude land plots with formal structures in 2001 building census
\end{enumerate}}\end{minipage}};

\onslide<4->\node[overlay,anchor=west,align=left] at (0, -.5) {    \begin{minipage}{1\textwidth} { 
\begin{enumerate}
  \setcounter{enumi}{3}
  \item \textbf{Spatial Clustering:} collect nearby houses into projects with density-based clustering algorithm
\end{enumerate}}\end{minipage}};

\onslide<5->\node[overlay,anchor=west,align=left] at (0, -1.5) {    \begin{minipage}{1\textwidth} { 
\begin{enumerate}
  \setcounter{enumi}{4}
  \item \textbf{Temporal Clustering:} include clusters with $>$50\% of transactions during modal year
\end{enumerate}}\end{minipage}};

\onslide<5->\node[overlay,anchor=west,align=left] at (0, -2.6) {    \begin{minipage}{1\textwidth} { 
\begin{itemize}
  \item Overlaps well with completed projects from admin. data
\end{itemize}}\end{minipage}};



\onslide<1>\node[overlay,anchor=west,align=left] at (0, -1) {
\begin{minipage}{1\textwidth} { 
\begin{figure}
\caption{Top 5 Seller Names}
\begin{tabu}{lc}
\toprule
 Seller Name & Observations \\
\midrule
City Of Johannesburg Metropolitan Municipality & 29,087  \\
City Of Johannesburg & 27,672  \\
City Of Tshwane Metropolitan Municipality & 24,780  \\
Ekurhuleni Metropolitan Municipality & 21,758  \\
Gauteng Provincial Housing Advisory Board & 13,058  \\
{\bf Total Observations }& {\bf 549,704}  \\
\bottomrule
\end{tabu}
 
\end{figure}
}\end{minipage}
} ;
% descriptive_statistics.do   program: write_biggest_sellers


\onslide<2>\node[overlay,anchor=west,align=left] at (0, -1.7) { 
\begin{minipage}{1\textwidth} { 
\begin{figure}
\caption{Purchase Price Densities}
 \includegraphics[scale=.53]{price_histogram.pdf} 
\end{figure}
}\end{minipage} 
} ;
% descriptive_statistics.do   program: write_price_histogram


\onslide<4>\node[overlay,anchor=west,align=left] at (2, -2.8) {  \includegraphics[scale=.16]{rdp_conhull_pic.png}  };
% generated from QGIS


\end{tikzpicture}

\end{frame}





%------------------------------------------------


\begin{frame}
\frametitle{Identifying Planned but Unconstructed Projects}

\begin{enumerate}
  \item Admin. data have ``planned,'' ``proposed,'' ``implementing'' projects
    \begin{itemize}
      \item Exclude projects with identified project transactions
    \end{itemize}

    \vspace{.2cm}

  \item Assign projects an expected completion date
    \begin{itemize}
      \item Fuzzy-string match budget data (with start-dates) on project names
      \item Add avg. diff. between transaction-date and start-date for completed projects
    \end{itemize}
\end{enumerate}

\begin{itemize}
  \item Why are projects canceled/delayed? 
    \begin{itemize}
      \item Legal disputes, service delivery backlogs, funding complications
      \item Delays often exceed 12 years 
    \end{itemize} 
\end{itemize}

\end{frame}


%------------------------------------------------

\begin{frame}
\frametitle{Housing Projects}
\begin{table}
\caption{Housing Projects and Building Growth}
\begin{tabu}{lcc}
 & Completed & Uncompleted \\ 
 Formal Density: 2001  & 338.3  & 238.9  \\ 
 Formal Density: 2011  & 1,718.4  & 691.6  \\ 
 &  &  \\ 
 Informal Density: 2001  & 435.2  & 2,125.8  \\ 
 Informal Density: 2011  & 1,022.9  & 2,950.4  \\ 
 &  &  \\ 
 Median Year (est.)  & 2005  & 2007  \\ 
 Distance to CBD (km)  & 29.7  & 29.8  \\ 
 &  &  \\ 
 Total Projects   & 57  & 64  \\ 
\bottomrule
\end{tabu}

\end{table}
\vspace{.2cm} 
Density is building number per square kilometer.
\end{frame}

%------------------------------------------------

\begin{frame}
\frametitle{Measure outcomes in close neighborhoods}
\begin{itemize}
  \item Focus on 1.2 km buffers around housing projects
\end{itemize}
\begin{center}
\begin{figure}
\includegraphics[scale=0.30]{design2.png}
\vspace{-3mm}
\end{figure}
\end{center}
\end{frame}

%------------------------------------------------

\begin{frame}
\frametitle{Housing Price Descriptives}
\begin{table}
\caption{Price Descriptives}
\centering
\resizebox{.95\textwidth}{!}{  
\begin{tabu}{lccccc}
\toprule
 & Completed Project       & Completed Buffer       & Uncompleted Project       & Uncompleted Buffer       & Other      \\
 &        & ($<$1.2 km)       &        & ($<$1.2 km)       &      \\
\midrule
 Purchase Price (Rand)  & 24,421.2  & 181,435.7  & 164,106.1  & 164,700.8  & 168,848.8  \\ 
\rowfont{\footnotesize} & [21,999.5]  & [149,871.1]  & [113,829.7]  & [147,593.1]  & [169,768.3]  \\ 
 &  &  &  &  &  \\ 
 Plot Size (m3)  & 280.9  & 389.7  & 299.4  & 469.0  & 772.0  \\ 
\rowfont{\footnotesize} & [127.7]  & [1,019.8]  & [592.2]  & [1,136.7]  & [2,606.0]  \\ 
 &  &  &  &  &  \\ 
 Sold At Least Once  & 0.136  & 0.413  & 0.582  & 0.537  & 0.319  \\ 
 Median Purchase Year  & 2005  & 2006  & 2007  & 2006  & 2006  \\ 
\midrule
 Observations  & 39,048  & 113,223  & 11,288  & 99,745  & 138,567  \\ 
\bottomrule
\end{tabu}

}
\end{table}
% descriptive_statistics.do program: write_descriptive_table
\end{frame}

%------------------------------------------------

\begin{frame}
\frametitle{Census Descriptives}

\centering
\resizebox{\textwidth}{!}{  
\begin{tabu}{lcccc}
 & \multicolumn{2}{c}{Within Project}     & \multicolumn{2}{c}{Outside Project}    \\
 & \multicolumn{2}{c}{($>$30\% Overlap)}  & \multicolumn{2}{c}{($<$30\% Overlap)}   \\
 &  &  &  &  \\ 
 & Completed & Uncompleted & Completed  & Uncompleted  \\
\midrule
 Flush Toilet  & 0.56  & 0.09  & 0.77  & 0.80  \\ 
 &  &  &  &  \\ 
 Piped Water  & 0.21  & 0.02  & 0.41  & 0.35  \\ 
 &  &  &  &  \\ 
 Elec. Cooking  & 0.58  & 0.25  & 0.68  & 0.71  \\ 
 &  &  &  &  \\ 
 Elec. Light  & 0.79  & 0.52  & 0.74  & 0.84  \\ 
 &  &  &  &  \\ 
 Single House  & 0.51  & 0.46  & 0.52  & 0.57  \\ 
 &  &  &  &  \\ 
\midrule
 Observations  & 59,460  & 16,529  & 213,061  & 49,286  \\ 
\bottomrule
\end{tabu}

}
% descriptive_statistics.do program: write_census_hh_table

\end{frame}


%------------------------------------------------

%% THIS SLIDE INCLUDES THE MATCHING METHOD %% 
\begin{frame}
\frametitle{Matching Method}

\centering
\begin{tabu}{lcccc}
 & \multicolumn{2}{c}{Unconstructed} & \multicolumn{2}{c}{Constructed} \\
 & Matched  & Unmatched  & Matched  & Unmatched  \\
\midrule
 Formal Density: 2001  & 188.4  & 142.5  & 481.7  & 546.1  \\ 
 Formal Density: 2011  & 505.0  & 284.0  & 1,527.2  & 1,575.4  \\ 
 &  &  &  &  \\ 
 Informal Density: 2001  & 998.6  & 457.2  & 1,611.1  & 985.2  \\ 
 Informal Density: 2011  & 1,619.1  & 701.3  & 2,379.3  & 1,064.2  \\ 
 &  &  &  &  \\ 
 Project House Density  & 0.0  & 0.0  & 558.6  & 348.4  \\ 
 Project Mode Year  & .  & .  & 2004  & 2005  \\ 
 &  &  &  &  \\ 
 Area (km2)  & 2.2  & 2.9  & 5.4  & 1.4  \\ 
 &  &  &  &  \\ 
\midrule
 Observations  & 67  & 32  & 75  & 18  \\ 
\bottomrule
\end{tabu}


\end{frame}





\end{document} 
