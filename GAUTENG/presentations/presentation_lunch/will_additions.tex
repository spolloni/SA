%%%%%%%%%%%%%%%%%%%%%%%%%%%%%%%%%%%%%%%%%
% Beamer Presentation
% LaTeX Template
% Version 1.0 (10/11/12)
%
% This template has been downloaded from:
% http://www.LaTeXTemplates.com
%
% License:
% CC BY-NC-SA 3.0 (http://creativecommons.org/licenses/by-nc-sa/3.0/)
%
%%%%%%%%%%%%%%%%%%%%%%%%%%%%%%%%%%%%%%%%%

%----------------------------------------------------------------------------------------
%	PACKAGES AND THEMES
%----------------------------------------------------------------------------------------

\documentclass[aspectratio=149]{beamer}
\usefonttheme[onlymath]{serif}


\mode<presentation> {

% The Beamer class comes with a number of default slide themes
% which change the colors and layouts of slides. Below this is a list
% of all the themes, uncomment each in turn to see what they look like.

\usetheme{default}
%\usetheme{AnnArbor}
%\usetheme{Antibes}
%\usetheme{Bergen}
%\usetheme{Berkeley}
%\usetheme{Berlin}
%\usetheme{Boadilla}
%\usetheme{CambridgeUS}
%\usetheme{Copenhagen}
%\usetheme{Darmstadt}
%\usetheme{Dresden}
%\usetheme{Frankfurt}
%\usetheme{Goettingen}
%\usetheme{Hannover}
%\usetheme{Ilmenau}
%\usetheme{JuanLesPins}
%\usetheme{Luebeck}
%\usetheme{Malmoe}
%\usetheme{Marburg}
%\usetheme{Montpellier}
%\usetheme{PaloAlto}
%\usetheme{Pittsburgh}
%\usetheme{Rochester}
%\usetheme{Singapore}
%\usetheme{Szeged}
%\usetheme{Warsaw}

% As well as themes, the Beamer class has a number of color themes
% for any slide theme. Uncomment each of these in turn to see how it
% changes the colors of your current slide theme.

%\usecolortheme{albatross}
\usecolortheme{beaver}
%\usecolortheme{beetle}
%\usecolortheme{crane}
%\usecolortheme{dolphin}
%\usecolortheme{dove}
%\usecolortheme{fly}
%\usecolortheme{lily}
%\usecolortheme{orchid}
%\usecolortheme{rose}
%\usecolortheme{seagull}
%\usecolortheme{seahorse}
%\usecolortheme{whale}
%\usecolortheme{wolverine}

%\setbeamertemplate{footline} % To remove the footer line in all slides uncomment this line
%\setbeamertemplate{footline}[page number] % To replace the footer line in all slides with a simple slide count uncomment this line

%\setbeamertemplate{navigation symbols}{} % To remove the navigation symbols from the bottom of all slides uncomment this line
}

\usepackage{graphicx} % Allows including images
\usepackage{booktabs} % Allows the use of \toprule, \midrule and \bottomrule in tables
\usepackage{verbatim}

\usepackage{mathtools} 
\usepackage{amssymb}
\usepackage{mathrsfs}
\usepackage{amsmath}
\usepackage{bm}

\usepackage{ragged2e}
\usepackage{etoolbox}
\usepackage{lipsum}

\usepackage{siunitx,booktabs}
\usepackage{pifont}
\usepackage{array}
\usepackage{tabu,booktabs}
\usepackage{tikz}
\usetikzlibrary{arrows,shapes}

\setbeamertemplate{enumerate items}[circle]
\usepackage{tikz}

\usepackage{dsfont}

\definecolor{manatee}{rgb}{0.59, 0.6, 0.67}
\definecolor{gr}{gray}{0.23}
\definecolor{navyblue}{rgb}{0.0, 0.0, 0.5}
\definecolor{lincolngreen}{rgb}{0.11, 0.35, 0.02}
\definecolor{copper}{rgb}{0.72, 0.45, 0.2}
\definecolor{darkorange}{rgb}{1.0, 0.55, 0.0}
\definecolor{darkpowderblue}{rgb}{0.0, 0.2, 0.6}
\definecolor{darksienna}{rgb}{0.24, 0.08, 0.08}
\definecolor{darkred}{rgb}{0.55, 0.0, 0.0}

\newcommand\mynum[1]{
  \usebeamercolor{enumerate item}
  \tikzset{beameritem/.style={circle,inner sep=0,minimum size=2ex,text=enumerate item.bg,fill=enumerate item.fg,font=\footnotesize}}%
  \tikz[baseline=(n.base)]\node(n)[beameritem]{#1};
}

\newcommand\mynumm[1]{
  \usebeamercolor{enumerate item}
  \tikzset{beameritem/.style={rectangle,inner sep=0,minimum size=2ex,text=enumerate item.bg,fill=enumerate item.fg,font=\footnotesize}}%
  \tikz[baseline=(n.base)]\node(n)[beameritem]{#1};
}

\def\Put(#1,#2)#3{\leavevmode\makebox(0,0){\put(#1,#2){#3}}}

\setbeamertemplate{footline}[]

%----------------------------------------------------------------------------------------
%	TITLE PAGE
%----------------------------------------------------------------------------------------

\title{Urban Water and Housing Infrastructure  \\ for Economic Development} % The short title appears at the bottom of every slide, the full title is only on the title page

\author{William Violette} 

 % Your institution as it will appear on the bottom of every slide, may be shorthand to save space

\date{May 2018} %\today} % Date, can be changed to a custom date

\begin{document}

\beamertemplatenavigationsymbolsempty

\begin{frame}
\titlepage % Print the title page as the first slide
\end{frame}




%%%% Date Selection %%%%

\begin{frame}
\frametitle{Expected Completion Dates for Uncompleted Projects}

\begin{itemize}
%  \item Admin data: 99 Uncompleted, 68 Completed Projects
  \item Treasury Budget Reports (2004-2009): 134 Projects
    \begin{itemize}
      \item Match 114 with over 60\% string similarity
      \item 44 names match exactly
    \end{itemize}
  \item Completed projects: 3 yrs from start to sale (on average)
  \item Uncompleted projects: expected sale yr = start yr + 3 yrs
    \begin{itemize}
      \item 99 total $\rightarrow$ 65 dated projects
    \end{itemize}
%      \item Assign random month
\end{itemize}
\end{frame}




%%%% Project Quotes %%%%

\begin{frame}
\frametitle{Why were some projects canceled?}
\begin{itemize}
\item ``R95m down the tubes as housing project picked apart brick by brick'' (Timeslive, 2017)
  \begin{itemize}
    \item Disputes over beneficiaries; disagreement with security contractor
  \end{itemize}
\vspace{.5cm}
\item ``MEC Mashatile delays Munsieville Ext 5 multimillion housing project'' (DA-GPL, 2017)
  \begin{itemize}
    \item Initiated in 2009; lacked approval from all agencies
  \end{itemize}
\vspace{.5cm}
\item ``Objections put R242m housing project on hold'' (IOL News, 2016)
  \begin{itemize}
    \item Failed environmental impact assessment
  \end{itemize}

\end{itemize}
\end{frame}




%%%% Project Status %%%%

\begin{frame}
\frametitle{Project Status}
\centering
\begin{table}
\caption{Project Status}
\begin{tabular}{l*{1}{cc}}
 &Unconstructed &Constructed  \\
\hline 
Implementation, Completed &          5 &         18  \\
Planning &         12 &          8  \\
Future, Investigating, Proposed &         18 &          5  \\
Uncertain &          2 &          0  \\
\hline No Description &         28 &         37  \\
\hline
\end{tabular}
\end{table}
%\vspace{.2cm}

\footnotesize{Source: Admin. project data} \\
\vspace{.1cm}
\footnotesize{Note: some projects have multiple \\ status keywords in their descriptions}
\end{frame}


%%%% Descriptive Table %%%%

\begin{frame}

\frametitle{Project Descriptives}

\centering
\begin{table}
\caption{Descriptives at Baseline}
\begin{tabular}{l*{1}{cc}}
 &Uncompleted &Completed  \\
\hline 
Number of Projects &         65 &         68  \\
Area (km2) &       2.21 &       3.62  \\
Project Houses (per km2) &  &        516  \\
Informal Buildings (per km2) &      1,194 &        997  \\
Formal Buildings (per km2) &        355 &        428  \\
House Price (Rand) &    113,956 &     88,781  \\
\hline
\end{tabular}
\end{table}

\end{frame}



\end{document} 
