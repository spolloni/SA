%%%%%%%%%%%%%%%%%%%%%%%%%%%%%%%%%%%%%%%%%
% Beamer Presentation
% LaTeX Template
% Version 1.0 (10/11/12)
%
% This template has been downloaded from:
% http://www.LaTeXTemplates.com
%
% License:
% CC BY-NC-SA 3.0 (http://creativecommons.org/licenses/by-nc-sa/3.0/)
%
%%%%%%%%%%%%%%%%%%%%%%%%%%%%%%%%%%%%%%%%%

%----------------------------------------------------------------------------------------
%	PACKAGES AND THEMES
%----------------------------------------------------------------------------------------

\documentclass[aspectratio=149]{beamer}
\usefonttheme[onlymath]{serif}


\mode<presentation> {

% The Beamer class comes with a number of default slide themes
% which change the colors and layouts of slides. Below this is a list
% of all the themes, uncomment each in turn to see what they look like.

\usetheme{default}
%\usetheme{AnnArbor}
%\usetheme{Antibes}
%\usetheme{Bergen}
%\usetheme{Berkeley}
%\usetheme{Berlin}
%\usetheme{Boadilla}
%\usetheme{CambridgeUS}
%\usetheme{Copenhagen}
%\usetheme{Darmstadt}
%\usetheme{Dresden}
%\usetheme{Frankfurt}
%\usetheme{Goettingen}
%\usetheme{Hannover}
%\usetheme{Ilmenau}
%\usetheme{JuanLesPins}
%\usetheme{Luebeck}
%\usetheme{Malmoe}
%\usetheme{Marburg}
%\usetheme{Montpellier}
%\usetheme{PaloAlto}
%\usetheme{Pittsburgh}
%\usetheme{Rochester}
%\usetheme{Singapore}
%\usetheme{Szeged}
%\usetheme{Warsaw}

% As well as themes, the Beamer class has a number of color themes
% for any slide theme. Uncomment each of these in turn to see how it
% changes the colors of your current slide theme.

%\usecolortheme{albatross}
\usecolortheme{beaver}
%\usecolortheme{beetle}
%\usecolortheme{crane}
%\usecolortheme{dolphin}
%\usecolortheme{dove}
%\usecolortheme{fly}
%\usecolortheme{lily}
%\usecolortheme{orchid}
%\usecolortheme{rose}
%\usecolortheme{seagull}
%\usecolortheme{seahorse}
%\usecolortheme{whale}
%\usecolortheme{wolverine}

%\setbeamertemplate{footline} % To remove the footer line in all slides uncomment this line
%\setbeamertemplate{footline}[page number] % To replace the footer line in all slides with a simple slide count uncomment this line

%\setbeamertemplate{navigation symbols}{} % To remove the navigation symbols from the bottom of all slides uncomment this line
}

\usepackage{graphicx} % Allows including images
\usepackage{booktabs} % Allows the use of \toprule, \midrule and \bottomrule in tables
\usepackage{verbatim}

\usepackage{mathtools} 
\usepackage{amssymb}
\usepackage{mathrsfs}
\usepackage{amsmath}
\usepackage{bm}

\usepackage{ragged2e}
\usepackage{etoolbox}
\usepackage{lipsum}

\usepackage{siunitx,booktabs}
\usepackage{pifont}
\usepackage{array}
\usepackage{tabu,booktabs}
\usepackage{tikz}
\usetikzlibrary{arrows,shapes}

\setbeamertemplate{enumerate items}[circle]
\usepackage{tikz}

\usepackage{dsfont}

\newcommand\mynum[1]{
  \usebeamercolor{enumerate item}
  \tikzset{beameritem/.style={circle,inner sep=0,minimum size=2ex,text=enumerate item.bg,fill=enumerate item.fg,font=\footnotesize}}%
  \tikz[baseline=(n.base)]\node(n)[beameritem]{#1};
}

\newcommand\mynumm[1]{
  \usebeamercolor{enumerate item}
  \tikzset{beameritem/.style={rectangle,inner sep=0,minimum size=2ex,text=enumerate item.bg,fill=enumerate item.fg,font=\footnotesize}}%
  \tikz[baseline=(n.base)]\node(n)[beameritem]{#1};
}

\def\Put(#1,#2)#3{\leavevmode\makebox(0,0){\put(#1,#2){#3}}}

\setbeamertemplate{footline}[]

%----------------------------------------------------------------------------------------
%	TITLE PAGE
%----------------------------------------------------------------------------------------

\title{ Subsidized Housing with Slum Externalities:  Evidence from South Africa } % The short title appears at the bottom of every slide, the full title is only on the title page

\author{Stefano Polloni 
joint with Ben Bradlow and Will Violette} 

 % Your institution as it will appear on the bottom of every slide, may be shorthand to save space

\date{February 2017} %\today} % Date, can be changed to a custom date

\begin{document}

\beamertemplatenavigationsymbolsempty


%\begin{frame}
%\frametitle{Overview} % Table of contents slide, comment this block out to remove it
%\tableofcontents % Throughout your presentation, if you choose to use \section{} and \subsection{} commands, these will automatically be printed on this slide as an overview of your presentation
%\end{frame}


%------------------------------------------------

%\begin{frame}
%\frametitle{Slums and Development}
%\begin{tabular}{lc} \hline
 & (1) \\
VARIABLES & Log Price \\ \hline
 &  \\
3 yrs 0-400m & -0.125 \\
 & (0.0892) \\
3 yrs 0-400m X In-Situ & 0.180 \\
 & (0.289) \\
 &  \\
Observations & 28,701 \\
R-squared & 0.489 \\
Project FE & YES \\
 Year-Month FE & YES \\ \hline
\multicolumn{2}{c}{ Robust standard errors in parentheses} \\
\multicolumn{2}{c}{ *** p$<$0.01, ** p$<$0.05, * p$<$0.1} \\
\multicolumn{2}{c}{ All control for cubic in plot size. Standard errors are clustered at the project level.} \\
\end{tabular}

%\end{frame}




%----------------------------------------------------------------------------------------
%	PRESENTATION SLIDES
%----------------------------------------------------------------------------------------
\section{Introduction}
%------------------------------------------------




\begin{frame}
\frametitle{Public Housing and Development}

\resizebox{\textwidth}{!}{  

\begin{tabular}{lcccccccccc} \hline
 & (1) & (2) & (3) & (4) & (5) & (6) & (7) & (8) & (9) & (10) \\
VARIABLES & Flush Toilet & Piped Water Inside & Electric Cooking & Electric Lighting & Single House & Owns House & No. Rooms & Household Size & Households per m2 & People per m2 \\ \hline
 &  &  &  &  &  &  &  &  &  &  \\
spillover & 0.00206 & -0.0758*** & -0.0190 & -0.0276 & -0.0275 & -0.00471 & -0.218*** & 0.0917* & 0.000160 & 0.000695 \\
 & (0.0222) & (0.0206) & (0.0247) & (0.0283) & (0.0296) & (0.0224) & (0.0665) & (0.0494) & (0.000321) & (0.000851) \\
project & -0.211*** & -0.130*** & -0.269*** & -0.272*** & -0.208*** & -0.217*** & -0.936*** & -0.516*** & 0.000856* & 0.000834 \\
 & (0.0397) & (0.0412) & (0.0402) & (0.0396) & (0.0414) & (0.0319) & (0.127) & (0.0836) & (0.000479) & (0.000884) \\
spillover\_rdp & -0.00225 & -0.0228 & -0.00104 & 0.0293 & 0.0427 & 0.0149 & -0.0935 & 0.131** & 0.000569 & 0.00205 \\
 & (0.0354) & (0.0373) & (0.0318) & (0.0323) & (0.0396) & (0.0282) & (0.143) & (0.0623) & (0.000521) & (0.00161) \\
project\_rdp & -0.0185 & -0.0826 & -0.0898 & 0.0231 & 0.0867 & 0.151*** & 0.0883 & 0.525*** & -0.000795 & -0.000298 \\
 & (0.0587) & (0.0525) & (0.0561) & (0.0550) & (0.0614) & (0.0432) & (0.149) & (0.115) & (0.000523) & (0.000996) \\
post & 0.0215 & 0.149*** & 0.0607*** & 0.0417*** & 0.0266** & 0.319*** & 0.347*** & -0.186*** & -0.000164 & -0.000651 \\
 & (0.0144) & (0.0121) & (0.0152) & (0.0135) & (0.0134) & (0.0374) & (0.0416) & (0.0299) & (0.000321) & (0.000735) \\
spillover\_post & 0.0172 & 0.0336** & 0.0405** & 0.0282 & 0.0297* & -0.0102 & -0.0959** & -0.0936*** & 0.000577* & 0.00126* \\
 & (0.0179) & (0.0150) & (0.0168) & (0.0191) & (0.0157) & (0.0321) & (0.0445) & (0.0322) & (0.000295) & (0.000679) \\
project\_post & -0.0244 & -0.0897*** & 0.0376 & 0.0208 & 0.00433 & 0.00691 & -0.122 & 0.0490 & 0.00103** & 0.00253*** \\
 & (0.0282) & (0.0234) & (0.0323) & (0.0280) & (0.0278) & (0.0345) & (0.0920) & (0.0482) & (0.000474) & (0.000951) \\
spillover\_rdp\_post & 0.0236 & 0.00930 & 0.0248 & -0.0196 & 0.0239 & -0.0309 & -0.0187 & -0.0118 & -0.000262 & -0.000779 \\
 & (0.0225) & (0.0187) & (0.0240) & (0.0197) & (0.0232) & (0.0269) & (0.0647) & (0.0353) & (0.000349) & (0.000912) \\
project\_rdp\_post & 0.182*** & 0.145*** & 0.323*** & 0.184*** & 0.146*** & -0.0676 & 0.169 & -0.0914 & -0.000169 & -0.000462 \\
 & (0.0416) & (0.0344) & (0.0449) & (0.0412) & (0.0467) & (0.0461) & (0.153) & (0.0973) & (0.000690) & (0.00114) \\
Constant & 0.836*** & 0.497*** & 0.779*** & 0.851*** & 0.595*** & 0.542*** & 3.798*** & 3.300*** & 0.00210*** & 0.00652*** \\
 & (0.0217) & (0.0218) & (0.0249) & (0.0251) & (0.0250) & (0.0252) & (0.0874) & (0.0461) & (0.000276) & (0.000729) \\
 &  &  &  &  &  &  &  &  &  &  \\
Observations & 4,603,181 & 4,603,181 & 4,603,181 & 4,603,181 & 4,374,511 & 4,455,014 & 4,283,738 & 4,567,545 & 21,923 & 21,923 \\
R-squared & 0.282 & 0.192 & 0.221 & 0.216 & 0.126 & 0.153 & 0.151 & 0.055 & 0.274 & 0.288 \\
 Project FE & YES & YES & YES & YES & YES & YES & YES & YES & YES & YES \\ \hline
\multicolumn{11}{c}{ Robust standard errors in parentheses} \\
\multicolumn{11}{c}{ *** p$<$0.01, ** p$<$0.05, * p$<$0.1} \\
\multicolumn{11}{c}{ Standard errors are clustered at the project level.} \\
\end{tabular}


}

\end{frame}




\end{document} 
