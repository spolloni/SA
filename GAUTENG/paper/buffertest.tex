\documentclass[12pt]{article}

% TEMPLATE DEFAULT PACKAGES
\usepackage{amssymb,amsmath,amsfonts,eurosym,geometry,ulem,graphicx,color,setspace,sectsty,comment,natbib,pdflscape,array,adjustbox}

% ADDED PACKAGES FOR THIS MANUSCRIPT
\usepackage{palatino,newtxmath,multirow,titlesec,threeparttable,tabu,booktabs,titlesec,threeparttable,mathtools,bm,bbm,subcaption,pdflscape,tcolorbox,mathrsfs}
% endfloat,

\usepackage{afterpage}
\usepackage[hyphens]{url}
\usepackage[margin=1cm]{caption}

\usepackage[draft]{hyperref}
\newcommand{\tim}{$\,\times\,$}
% FIGURES & TABLES CAPTION STYLING
\captionsetup[figure]{labelfont={bf},name={Figure},labelsep=period}
\captionsetup[table]{labelfont={bf},name={Table},labelsep=period}

% SECTION TITLE SETTINGS
\titlelabel{\thetitle.\enskip}
\titleformat*{\section}{\large\bfseries}
\titleformat*{\subsection}{\normalsize\bfseries}

% COLUMN TYPES
\newcolumntype{L}[1]{>{\raggedright\let\newline\\\arraybackslash\hspace{0pt}}m{#1}}
\newcolumntype{C}{>{\centering\arraybackslash}p{5.2em}}
\newcolumntype{D}{>{\centering\arraybackslash}p{5em}}
\newcolumntype{R}[1]{>{\raggedleft\let\newline\\\arraybackslash\hspace{0pt}}m{#1}}


% MARGINS AND SPACING
\normalem
\geometry{left=1.1in,right=1.1in,top=1.0in,bottom=1.0in}
\setlength{\parskip}{2.5pt}

% SPECIAL CELL 
\newcommand{\specialcell}[2][c]{%
	\begin{tabular}[#1]{@{}l@{}}#2\end{tabular}}

% NO INDENT ON FOOTNOTES
\usepackage[hang,flushmargin]{footmisc}

\begin{document}

\begin{figure}
\centering
\includegraphics[scale=.55]{figures/buffertest1.pdf}
\end{figure}

\begin{align*}
Y_{it} &= \beta_0 + \beta_1 post_t + \beta_1 project_i + \beta_2 buffer_i + \beta_3 project_i \times post_t + \beta_4 buffer_i \times post_t + \epsilon_{it} \\
\text{Where }& \\
i & = \text{census shape} \\
t & = \text{pre/post} \\
Y_{it} & = \text{census characteristics in shape $i$ at time $t$} \\
project_i & = \frac{\text{Area(Intersection(housing project, census shape))}}{\text{Area(census shape)}} \\
buffer_i & = \frac{\text{Area(Intersection(buffer, census shape))}}{\text{Area(census shape)}} 
\end{align*}

\end{document}


