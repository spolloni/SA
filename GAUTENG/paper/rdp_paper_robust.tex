\documentclass[12pt]{article}

% TEMPLATE DEFAULT PACKAGES
\usepackage{amssymb,amsmath,amsfonts,eurosym,geometry,ulem,graphicx,color,setspace,sectsty,comment,natbib,pdflscape,array,adjustbox}

% ADDED PACKAGES FOR THIS MANUSCRIPT
\usepackage{palatino,newtxmath,multirow,titlesec,threeparttable,tabu,booktabs,titlesec,threeparttable,mathtools,bm,bbm,subcaption,pdflscape,tcolorbox,mathrsfs,tikz,float,longtable}
% endfloat,
% \usepackage[capposition=top]{floatrow}
\usepackage{tikzpagenodes}
\usepackage{pbox}

\usepackage{afterpage}
\usepackage[hyphens]{url}
\usepackage[margin=1cm]{caption}

\usepackage[draft]{hyperref}
\newcommand{\tim}{$\,\times\,$}
% FIGURES & TABLES CAPTION STYLING
\captionsetup[figure]{labelfont={bf},name={Figure},labelsep=period}
\captionsetup[table]{labelfont={bf},name={Table},labelsep=period}

% SECTION TITLE SETTINGS
\titlelabel{\thetitle.\enskip}
\titleformat*{\section}{\large\bfseries}
\titleformat*{\subsection}{\normalsize\bfseries}

% COLUMN TYPES
\newcolumntype{L}[1]{>{\raggedright\let\newline\\\arraybackslash\hspace{0pt}}m{#1}}
\newcolumntype{C}{>{\centering\arraybackslash}p{5.2em}}
\newcolumntype{D}{>{\centering\arraybackslash}p{5em}}
\newcolumntype{G}{>{\centering\arraybackslash}p{6em}}
\newcolumntype{R}[1]{>{\raggedleft\let\newline\\\arraybackslash\hspace{0pt}}m{#1}}


% MARGINS AND SPACING
\normalem
\geometry{left=1.1in,right=1.1in,top=1.0in,bottom=1.0in}
\setlength{\parskip}{2.5pt}

% SPECIAL CELL 
\newcommand{\specialcell}[2][c]{%
	\begin{tabular}[#1]{@{}l@{}}#2\end{tabular}}

% NO INDENT ON FOOTNOTES
\usepackage[hang,flushmargin]{footmisc}


\newcommand{\regtextfirst}{
	``hm'' refers to a hectometer or 100 meters. Observations are one hectare plots ($\text{hm}^{2}$).
``Plot footprint'' is the effect of an additional hectare of overlap with constructed projects.  
For plots outside of projects, ``Plot neighborhood'' is the effect of an additional hectare of overlap between 0 to 5 hm rings around plots and constructed projects.  Specifications also include separate effects for 5 hm rings from 0 to 40 hm.
Standard errors are clustered at the project level (in parentheses). 
\textsuperscript{c} p$<$0.10,\textsuperscript{b} p$<$0.05,\textsuperscript{a} p$<$0.01 \,\,
}

\newcommand{\regtext}{
	``hm'' refers to a hectometer or 100 meters. Observations are one hectare plots ($\text{hm}^{2}$).
``Plot footprint'' is the effect of an additional hectare of overlap with constructed projects.  
For plots outside of projects, ``Plot neighborhood'' is the effect of an additional hectare of overlap between 0 to 5 hm rings around plots and constructed projects.  Although not displayed in the table, specifications also include separate effects for 5 hm rings from 0 to 40 hm and results are available upon request.
Standard errors are clustered at the project level (in parentheses). 
\textsuperscript{c} p$<$0.10,\textsuperscript{b} p$<$0.05,\textsuperscript{a} p$<$0.01 \,\,
}

\newcommand{\ballparktext}{
	``Footprint'' sums the footprint effect across plots within constructed projects.
	``Spillover'' sums the neighborhood effect (0-.5 km) across plots nearby constructed projects.
	``Total'' sums ``Footprint'' and ``Spillover'' effects.
}

\newcommand{\hmref}{
	``hm'' refers to a hectometer or 100 meters.
}
\newcommand{\haref}{
	``ha'' refers to a hectare or one $\text{hm}^{2}$.
}

\newcommand{\hmrefha}{
	``hm'' refers to a hectometer or 100 meters. ``ha'' refers to a hectare or one $\text{hm}^{2}$.
}




\begin{document}

\begin{table}
\small
\centering
\caption{Effects on House Quality (2001-2011)}\label{table:inf_census}
\vspace{-2mm}
\resizebox{1\linewidth}{!}{
\begin{threeparttable}
\begin{tabular}{lCCCCC}
\toprule
                    &(1)&(2)&(3)&(4)&(5)\\[.5em] &Total Rooms                   &   Own House                   &Electric Lighting                   &Flush Toilet                   &Piped Water Inside\\ \midrule \\[-.6em]                   \\
Post $\times$ Constructed project overlap with: \\[1em] \hspace{1.5em}Plot footprint&      0.3492                   &      0.0396                   &      0.1185\textsuperscript{b}&      0.2367\textsuperscript{a}&      0.2795\textsuperscript{a}\\
                    &    (0.2950)                   &    (0.0535)                   &    (0.0555)                   &    (0.0591)                   &    (0.0550)                   \\[.5em]
\hspace{1.5em}Plot neighborhood (0-5 hm ring)&      0.0044                   &      0.0006                   &      0.0004                   &      0.0008                   &      0.0001                   \\
                    &    (0.0028)                   &    (0.0007)                   &    (0.0009)                   &    (0.0007)                   &    (0.0007)                   \\[.5em]
Mean Pre            &      3.5005                   &      0.3478                   &      0.6921                   &      0.6988                   &      0.4091                   \\
Mean Post           &      3.9838                   &      0.3507                   &      0.7696                   &      0.7515                   &      0.5816                   \\
R$^2$               &       0.063                   &       0.010                   &       0.049                   &       0.034                   &       0.102                   \\
N                   &     698,762                   &     699,801                   &     701,296                   &     701,296                   &     701,296                   \\

\bottomrule
\end{tabular}
\begin{tablenotes}
\item 
\end{tablenotes}
\end{threeparttable}
}
\end{table}

\begin{table}
\small
\centering
\caption{Effects on House Quality Placebo (1996-2001)}\label{table:inf_census_placebo}
\vspace{-2mm}
\resizebox{1\linewidth}{!}{
\begin{threeparttable}
\begin{tabular}{lCCCCC}
\toprule
                    &(1)&(2)&(3)&(4)&(5)\\[.5em] &Total Rooms                   &   Own House                   &Electric Lighting                   &Flush Toilet                   &Piped Water Inside\\ \midrule \\[-.6em]                   \\
Post $\times$ Constructed project overlap with: \\[1em] \hspace{1.5em}Plot footprint&      0.2947                   &     -0.1670\textsuperscript{b}&      0.1434                   &      0.1055                   &      0.0112                   \\
                    &    (0.4220)                   &    (0.0698)                   &    (0.1066)                   &    (0.1112)                   &    (0.0720)                   \\[.5em]
\hspace{1.5em}Plot neighborhood (0-5 hm ring)&     -0.0037                   &     -0.0024\textsuperscript{b}&      0.0011                   &     -0.0001                   &      0.0006                   \\
                    &    (0.0035)                   &    (0.0010)                   &    (0.0010)                   &    (0.0009)                   &    (0.0010)                   \\[.5em]
Mean Pre            &      3.6572                   &      0.4258                   &      0.6580                   &      0.6616                   &      0.5288                   \\
Mean Post           &      3.5005                   &      0.3478                   &      0.6921                   &      0.6988                   &      0.4091                   \\
R$^2$               &       0.044                   &       0.041                   &       0.037                   &       0.022                   &       0.064                   \\
N                   &     711,670                   &     717,899                   &     717,800                   &     717,800                   &     717,800                   \\

\bottomrule
\end{tabular}
\begin{tablenotes}
\item 
\end{tablenotes}
\end{threeparttable}
}
\end{table}



\begin{table}
\small
\centering
\caption{Effects on Census Outcomes (2001-2011)}\label{table:demo}
\vspace{-2mm}
\resizebox{1\linewidth}{!}{
\begin{threeparttable}
\begin{tabular}{lCCCCC}
\toprule
                    &(1)&(2)&(3)\\[.5em] &Formal                   &Single House                   &         Age                   &Household Size \\ \midrule \\[-.6em]                   \\
Post $\times$ Constructed project overlap with: \\[1em] \hspace{1.5em}Plot footprint&      0.2594\textsuperscript{a}&      0.2007\textsuperscript{a}&     -0.8989                   &      0.3078\textsuperscript{b}\\
                    &    (0.0570)                   &    (0.0689)                   &    (0.9067)                   &    (0.1498)                   \\[.5em]
\hspace{1.5em}Plot neighborhood (0-5 hm ring)&      0.0012\textsuperscript{c}&      0.0009                   &      0.0041                   &      0.0027\textsuperscript{c}\\
                    &    (0.0007)                   &    (0.0007)                   &    (0.0097)                   &    (0.0015)                   \\[.5em]
Mean Pre            &      0.5956                   &      0.5840                   &     41.5897                   &      3.0403                   \\
Mean Post           &      0.6653                   &      0.6734                   &     42.5959                   &      2.7759                   \\
R$^2$               &       0.041                   &       0.051                   &       0.031                   &       0.059                   \\
N                   &     701,395                   &     699,143                   &     699,656                   &     700,795                   \\

\bottomrule
\end{tabular}
\begin{tablenotes}
\item 
\end{tablenotes}
\end{threeparttable}
}
\end{table}



\begin{table}
\small
\centering
\caption{Effects on Census Outcomes Placebo (1996-2001)}\label{table:demo_placebo}
\vspace{-2mm}
\resizebox{1\linewidth}{!}{
\begin{threeparttable}
\begin{tabular}{lCCCCC}
\toprule
                    &(6)&(7)&(8)&(9)&(10)\\[.5em] &People                   &         Age                   &     Married                   &     African                   &Household Size \\ \midrule \\[-.6em]                   \\
Post $\times$ Constructed project overlap with: \\[1em] \hspace{1.5em}Plot footprint&      4.4899\textsuperscript{c}&      0.0442                   &     -0.0118                   &     -0.0248                   &      0.0426                   \\
                    &    (2.7055)                   &    (1.4060)                   &    (0.0294)                   &    (0.0591)                   &    (0.1928)                   \\[.5em]
\hspace{1.5em}Plot neighborhood (0-5 hm ring)&      0.0066                   &     -0.0119                   &     -0.0000                   &      0.0000                   &     -0.0042\textsuperscript{c}\\
                    &    (0.0537)                   &    (0.0120)                   &    (0.0003)                   &    (0.0006)                   &    (0.0025)                   \\[.5em]
Mean Pre            &     10.8966                   &     41.5792                   &      0.3589                   &      0.6283                   &      3.1617                   \\
Mean Post           &     13.4201                   &     41.5897                   &      0.3079                   &      0.7071                   &      3.0403                   \\
R$^2$               &       0.048                   &       0.029                   &       0.087                   &       0.102                   &       0.024                   \\
N                   &     717,899                   &     711,593                   &     717,899                   &     717,899                   &     712,422                   \\

\bottomrule
\end{tabular}
\begin{tablenotes}
\item 
\end{tablenotes}
\end{threeparttable}
}
\end{table}

\end{document}


