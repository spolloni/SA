\documentclass[12pt]{article}

% TEMPLATE DEFAULT PACKAGES
\usepackage{amssymb,amsmath,amsfonts,eurosym,geometry,ulem,graphicx,caption,color,setspace,sectsty,comment,footmisc,caption,natbib,pdflscape,array,hyperref,adjustbox}

% ADDED PACKAGES FOR THIS MANUSCRIPT
\usepackage{mathptmx,multirow,titlesec,threeparttable,tabu,booktabs,titlesec,subfigure,threeparttable,mathtools,bm,bbm}
% endfloat,
\usepackage{rotating}

% SECTION TITLE SETTINGS
\titlelabel{\thetitle.\enskip}
\titleformat*{\section}{\large\bfseries}
\titleformat*{\subsection}{\normalsize\bfseries}

% COLUMN TYPES
\newcolumntype{L}[1]{>{\raggedright\let\newline\\\arraybackslash\hspace{0pt}}m{#1}}
\newcolumntype{C}[1]{>{\centering\let\newline\\\arraybackslash\hspace{0pt}}m{#1}}
\newcolumntype{R}[1]{>{\raggedleft\let\newline\\\arraybackslash\hspace{0pt}}m{#1}}

% MARGINS AND SPACING
\normalem
%\onehalfspacing
\geometry{left=1in,right=1in,top=1.0in,bottom=1.0in}

% SPECIAL CELL 
\newcommand{\specialcell}[2][c]{%
	\begin{tabular}[#1]{@{}l@{}}#2\end{tabular}}

\begin{document}{}

\section*{ Results Summary }

\begin{itemize}
	\item House Quality and Infrastructure
		\begin{itemize}
			\item BIG upgrades in project (*)
			\item small upgrades in spillover (no *)
		\end{itemize}

	\item Population Density
		\begin{itemize}
			\item small increase in project (no *)
			\item Medium increase in spillover (no *)
		\end{itemize}

	\item Housing density
		\begin{itemize}
			\item Medium increase in project (no *)
				\begin{itemize}
					\item big shift for formal
				\end{itemize}
			\item Medium increase in spillover (no *)
				\begin{itemize}
					\item same composition
				\end{itemize}
		\end{itemize}

	\item Prices
		\begin{itemize}
			\item Big and negative really close (no *)
			\item Zero further (no *)
		\end{itemize}		

	\item Population 
		\begin{itemize}
			\item both project and spill : Better edu/employed/income people (*)
		\end{itemize}
\end{itemize}



\section*{ Pitch }

The way were thinking about welfare has three components what's happening (1) in the project area, (2) in the spillover area, and (3) in all other areas.  We find two interpretations depending on your statistical tolerance.
\begin{enumerate}
\item If you care about *'s (statistical significance) :
\begin{itemize}
	\item Zero price/housing spillovers combined with zero population increase means that all welfare consequences can be confined to impacts in project areas
	\item Assuming no externalities from population sorting (ie. high-skilled people moving around)
\end{itemize}
\item If you don't care about *'s (economic significance) :
\begin{itemize}
	\item We find positive spillovers nearby (new construction, population influx) which is evidence of a positive neighborhood externality
	\item This externality induces high-skilled people to move into the area  
		\begin{itemize}
			\item (probably aren't leaving slums so not reducing potential slum externalities in all other areas)
		\end{itemize}
	\item Residents of other areas likely benefit as rents decrease with fewer people there (Kline and Moretti) (assuming no additional externalities from population sorting)
\end{itemize}
\end{enumerate}



\section*{Key points}

\begin{itemize}
	\item First : positive welfare effects within projects (that's good)
		\begin{itemize}
			\item buildings get better, more formal structures
		\end{itemize}
	\item Second : we can mostly rule out localized spillover effects
		\begin{itemize}
			\item No nearby building construction/price gradients
			\item No improvements in nearby house quality
			\item No differential population sorting/growth
		\end{itemize}
	\item Third : we can't really rule out big (project+spillover) amenity boost (my interpretation of Jesse's comments is that he is hinting at this one)
		\begin{itemize}
			\item (project+spillover) composition improvement and population growth
		\end{itemize}
\end{itemize}

\subsection*{Interpreting a Big (project+spillover) amenity boost}

\begin{itemize}
	\item People move in!
		\begin{itemize}
			\item Kline and Moretti population movement distortions (wages and rents adjust to inefficient levels both here and elsewhere; should we sign those distortions ie. are far away places worse off?  Are there likely to be fewer slums in far away places now?)
			\item But : could be offset by positive public good provision (hard for us to say)
		\end{itemize}
	\item Smarter, more employed people move in!
		\begin{itemize}
			\item Definitely second-order, but could matter in a Diamond world with education specific agglomeration (maybe say why we don't think that's relevant in South Africa)
		\end{itemize}
\end{itemize}

\section*{Kline and Moretti}

What do people care about in a location?

\begin{itemize}
	\item Wage
	\item Rent
	\item Amenity
	\item Taste
\end{itemize}

\noindent Welfare effects (from tax increase in one city)

\begin{itemize}
	\item Cost of living increase (rents are out of whack)
	\item Dead weight loss (wages are out of whack)
	\item Goes to zero if nobody moves! (moving=evidence of distortion!)
\end{itemize}

\noindent Additional welfare effects

\begin{itemize}
	\item Local public goods
	\item Agglomeration in production/consumption 
	\item Unemployment and labor market frictions
	\item Credit constraints and missing insurance
\end{itemize}





\end{document}
