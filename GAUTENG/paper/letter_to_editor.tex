\documentclass{article}
\usepackage[letterpaper, margin=1in]{geometry}
\usepackage[utf8]{inputenc} % make more characters printable when copy-pasted here
\usepackage{xcolor} % to facilitate the myNotes and myTodo commands.
% \usepackage[dvipsnames]{xcolor}

\usepackage{hyperref}
\usepackage{xr-hyper}
\externaldocument{rdp_paper}
% Created by Jack Baker, last modified 11/3/2020

% initialize counters for reviewer number and comment number
\newcounter{reviewer}
\setcounter{reviewer}{0}
\newcounter{point}[reviewer]
\setcounter{point}{0}

\definecolor{replyblue}{RGB}{0, 30, 102}
% Set the format of reviewer comment numbers and responses.
\renewcommand{\thepoint}{Q\,\thereviewer.\arabic{point}} 
\newcommand{\point}[1]{\refstepcounter{point}  \bigskip \hrule \medskip \noindent {{\fontseries{b}\selectfont \thepoint } #1}\par}
% \newcommand{\reply}{\medskip \noindent \textbf{Reply}: \textcolor{blue} }  
\newcommand{\reply}{\medskip \noindent \textcolor{replyblue}} 
\newcommand{\ri}{\textcolor{replyblue} }  

\newcommand{\sr}{}
\newcommand{\er}{}
% \newcommand{\sr}{\begin{minipage}{\dimexpr\textwidth-3cm}}
% \newcommand{\er}{\end{minipage}}
% \newcommand{\cc{}{\medskip \noindent \textbf{Comment:}\hspace{2em} \textit}

\newcommand{\cc}{\noindent}
% Set the format for reviewer headings
\newcommand{\reviewersection}{\stepcounter{reviewer}
                  \section*{Reviewer \thereviewer}}

% Set the format for notes on in-progress responses
\newcommand\myNotes[1]{\textcolor{red}{#1}}
\newcommand\myTodo[1]{\textcolor{blue}{#1}}

%%%%%%%%%%%%%%%%%%%%%%%%%%%%%%%%%%%%%%%%%%%%%%%%%%%%%%%%
\title{Letter to the Editor}
\author{Benjamin Bradlow, Stefano Polloni, William Violette}
\begin{document}

\maketitle



\noindent Dear Donald R. Davis,
\vspace{1em}

Thank you for the opportunity to revise our paper for consideration at the Journal of Urban Economics.  Your suggestions as well as those of the referees have been very helpful in strengthening and adding clarity to the paper.  We include ``rdp\_paper\_revised\_edits.pdf'' which uses bold text to highlight edited text.  We address a few main points in bullets and include detailed responses (in blue) to the referees in ``response\_to\_reviews.pdf''.  

\begin{itemize}
\item Following your advice and comments from the referees, we have removed the welfare section and included a brief discussion of limitations to quantifying welfare (in the conclusion section).
\item We appreciate the referees' recommendations for addressing imbalance in pre-period observables and using observables for project selection, which we address with the following robustness exercises: (1) we update our propensity-score approach to include all pre-period observables to generate the score; (2) we include time-varying geographic controls; (3) we vary our definition of unconstructed projects; and (4) we exclude projects within 2 km of each other.  Results remain robust to these approaches.  See Robustness Section~\ref{section:robustness} for discussion.
\item We split the direct effect and spillover effect estimations in order to streamline the notation and exposition.  The results remain almost identical.
\item We normalized our project exposure measures so that the main coefficients can be interpreted simply as capturing the effect of an average project on a land plot.
\item We reorganized the results in terms of (1) measuring the projects (formal housing, infrastructure in project footprints), (2) estimating direct effects on informal housing in project footprints, and (3) estimating spillover effects.
\item We streamlined our discussion of potential mechanisms and used census results to favor some mechanisms over others.
\item We calculate average total new houses/population per project to be representative of projects (instead of projects weighted by area as before).
\end{itemize}
Please let us know if you have any questions, and we look forward to hearing your thoughts on the revisions.
\vspace{.1em}

\noindent Thanks again,
\vspace{1em}

Benjamin Bradlow, Stefano Polloni, William Violette



\end{document}



